\section*{Задание\\ на курсовую работу}

\setlength{\extrarowheight}{7mm}
\begin{tabularx}{\textwidth}{>{\raggedright\arraybackslash}X}
	Студент \student{}\\
	\group\\
	Тема работы: \theme \\
	Исходные данные: \ldots\\
	Содержание пояснительной записки: содержание, введение, \ldots, заключение, список использованных источников, \ldots\\
	Предполагаемый объём пояснительной записки: не менее 00 страниц.\\
	Дата выдачи задания: 00.00.0000\\
	Дата сдачи реферата: 00.00.0000\\
	Дата защиты реферата: 00.00.0000\\
\end{tabularx}
\setlength{\extrarowheight}{0mm}

\vspace{50mm}
%\vfill

\setlength{\extrarowheight}{4mm}
       \begin{tabulary}{\textwidth}{LCCCL}
            Студент & \hspace{0.5cm} & \hspace{4.5cm} & \hspace{0.5cm} & \student \\
            \cline{3-3}
            Преподаватель & \hspace{0.5cm} & \hspace{4.5cm} & \hspace{0.5cm} & \teacher \\
            \cline{3-3}
       \end{tabulary}
       \setlength{\extrarowheight}{0mm}

\newpage

\section*{АННОТАЦИЯ}
\begin{minipage}[t][0.4\textheight][t]{0.9\linewidth}
	\setlength{\parindent}{1.25cm}
	\indent
	Кратко (в 8--10 строк) указать основное содержание курсового проекта (курсовой работы), методы исследования (разработки), полученные результаты.
\end{minipage}

\selectlanguage{english}
\section*{SUMMARY}
\begin{minipage}[t][0.4\textheight][t]{0.9\linewidth}
	\setlength{\parindent}{1.25cm}
	\indent
	Summary in English.
\end{minipage}
\selectlanguage{russian}

\newpage

\let \savenumberline \numberline
\def \numberline#1{\savenumberline{#1.}}

\tableofcontents

\newpage

\section*{Введение}
\addcontentsline{toc}{section}{Введение}

Целью работы является \ldots\ Для достижения цели необходимо решить следующие задачи:

\begin{enumerate}
	\item \ldots;
	\item \ldots;
\end{enumerate}

\newpage